\chapter{Infrastruktur}

  \section{Bessere Vernetzung von Fahrradwegen im ländlichen Raum}
  
  Inzwischen sind im Landkreis Augsburg etliche Fahrradweg entlang der 
  Ortsverbindungsstraßen entstanden. Leider enden diese Fahrradwege oft direkt 
  am Ortsanfang oder kurz danach, ohne die Radfahrer wieder in den allgemeinen 
  Verkehrsfluss zu integrieren. Typisches Beispiel sind etwa gemeinsam 
  genutzte Rad- und Fußwege, die auf einmal zu reinen Fußwegen umgewidmet 
  werden, ohne dass diese eine Möglichkeit zur Ausleitung auf die Straße 
  erhalten. Radfahrer machen einen großen Teil des aktiven Verkehrs innerhalb 
  von Ortschaften aus. Trotzdem sind sie bisher kaum sinnvoll in bestehende 
  Verkehrskonzepte eingebunden.
  
  Die Piratenpartei setzt sich dafür ein, dass Fahrradwege durch Gemeinden 
  ausgewiesen und gebaut werden. Wenn eine Routenführung der Durchgangsstraße 
  entlang nicht möglich ist, sollten Alternativrouten über Seitenstraßen oder 
  um die Gemeinde herum entsprechend ausgebaut und beschildert werden.
  
  Vor allem bei Neu- und Umbauten von Ortsdurchfahrten muss auf einen 
  fahrradgerechten Ausbau geachtet werden. Hier hält sich die finanzielle 
  Mehrbelastung durch frühzeitige Planungsmöglichkeiten in Grenzen; es wird 
  zudem ein integriertes Verkehrskonzept ermöglicht, das verschiedenartige 
  Verkehrsteilnehmer wie Radfahrer, Fußgänger und Autofahrer gleichermaßen 
  berücksichtigt.
  
  Auch beim Neubau von Ortsumgehungsstraßen sollten zusätzlich Fahrradwege 
  entlang der Umgehung angelegt werden.
  
  \section{Öffentlicher Nahverkehr}
  
  Der Öffentliche Personennahverkehr (ÖPNV) spielt eine zentrale Rolle für die 
  Entwicklung einer Stadt und für die Versorgungsqualität der städtischen 
  Einrichtungen für die Bürger. Augsburg mit seiner dichtbebauten, kompakten 
  und straßenarmen Innenstadt ist als Versorgungsmittelpunkt Nordschwabens 
  anhaltend verkehrsüberlastet.
  
  Auch die kostspieligen und kontrovers aufgenommenen Straßenbaumaßnahmen 
  der letzten Jahre konnten den Individualverkehr nur teilweise neu ordnen, 
  ohne ihn als zentrales Mittel der Mobilität zu stark einzuschränken. Die 
  Kosten dieser und weiterer Straßenbaumaßnahmen belasten zusätzlich den 
  städtischen Haushalt und schränkt die zukünftige Gestaltungsfähigkeit 
  Augsburgs ein. Mit Verkehrsregeln und Verboten allein kann man den Bürgern 
  nicht zu städteplanerischer Weitsicht bewegen: Man muss ihnen ein besseres 
  Angebot machen.
  
  Glücklicherweise gibt es mit dem ÖPNV als Lösungsansatz für innerstädtische 
  Verkehrsprobleme bereits umfassende Erfahrungen. Die grundsätzliche 
  Wichtigkeit eines öffentlichen Transportnetzes steht heute außer Frage; um 
  es allerdings zur vollen Wirkung zu bringen, haben bereits mehrere Städte 
  erfolgreich die Finanzierung des Nahverkehrsnetzes neu definiert: Ebenso wie 
  Straßen, die dem Bürger kostenlos zur Verfügung stehen, entfaltet auch der 
  ÖPNV seine Wirkung als Instrument der Stadtentwicklung erst, wenn er 
  fahrscheinlos gemacht wurde. Fahrscheinlos bedeutet nicht kostenlos. Die 
  Finanzierung erfolgt über ein Umlageverfahren:\\ Durch den Wegfall des 
  Vertriebs der Fahrscheine kommt es zu Einsparungen, während die signifikant 
  ansteigenden Umsätzen bei Gastronomie und Einzelhandel im Innenstadtbereich 
  mit mehr Einnahmen generieren.
  
  Die Piratenpartei Augsburg schlägt zur Finanzierung das Model 2 x 15 vor. 
  Für die Finanzierung des fahrscheinlosen ÖPNV wird eine Gebühr in Höhe von 
  15€ je Monat pro Einwohner erhoben. Schüler, Studenten und sozial Schwache 
  sind von der Gebühr befreit. Die paritätische Beteiligung an der 
  Finanzierung der Unternehmen in Augsburg wird durch einen Beitrag von 15€ 
  pro Arbeitnehmer je Monat erreicht.
  
  Die Vorteile des fahrscheinlosen ÖPNV sind dagegen massiv:
  
  Wie schon in anderen Städten bringt ein für den Benutzer kostenfreier 
  Nahverkehr einen deutlichen Besucher-Zustrom in die Innenstadt: Handel und 
  Gastronomie blühen in Vergleichsszenarien um bis zu einem Drittel auf. Die 
  konstant verkehrsüberlastete Augsburger Innenstadt würde in großem Umfang 
  immissions-entlastet. Sowohl Lärm als auch Abgase werden durch stärkere 
  Nutzung des Nahverkehrs drastisch reduziert, was dem Nachholbedarf Augsburgs 
  beim Umweltschutz sehr entgegenkommt. Die freie Mobilität via ÖPNV entlastet 
  die sozialen Spannungen in der von Immigration geprägten nordschwäbischen 
  Metropole; Schüler, Studenten, Auszubildende, Sozial Schwache und Senioren 
  erhalten du"-rch den F-ÖPNV bessere Lebensqualität.
  
  Die Piratenpartei Augsburg will angesichts dieser eklatanten Vorteile den 
  fahrscheinlosen Öffentlichen Personennahverkehr in Augsburg einführen.
