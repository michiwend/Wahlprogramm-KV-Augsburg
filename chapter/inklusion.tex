\chapter{Inklusion}

  \section{Inklusion in Bildungseinrichtungen ermöglichen und erleichtern}
  
  Die Piratenpartei Augsburg setzt sich dafür ein, durch die Schaffung von 
  zusätzlichen Personalstellen die Inklusion von Kindern mit geistigen oder 
  körperlichen Einschränkungen zu ermöglichen. Des Weiteren sollen diese 
  Bemühungen um eine ganzheitliche Gesellschaft durch einfache aber effektive 
  bauliche Maßnahmen weiter erleichtert werden. Langfristiges Ziel ist die 
  Schaffung eines barrierefreien Zugangs zu allen städtischen 
  Bildungseinrichtungen.
  
  \section{Barrierefreiheit}
  
  Die Stadt Augsburg verpflichtet sich zur Einhaltung der UN-Konve"-ntion über 
  die Rechte von Menschen mit Behinderungen. Barrierefreiheit ist bei allen 
  städtischen Projekten bereits zu Beginn in den Planungsphasen zu 
  berücksichtigen. Alle Kreuzungen und Haltestellen sind zeitnah barrierefrei 
  mit abgesenkten Bordsteinen und taktilen Flächen für Blinde und sehbehinderte 
  Menschen zu versehen, Fußgängerampeln werden mit Audiosignalen nachgerüstet\\
  und regelmäßig gewartet. Die Innenstadt ist mittelfristig nach dem Vorbild 
  anderer Kommunen mit einem Leitsystem für Blinde und Sehbehinderte 
  auszustatten. Das Leitsystem soll in enger Zusammenarbeit mit den 
  Behindertenverbänden, sowie unter Koordination durch den Behindertenbeirat 
  der Stadt Augsburg erarbeitet und auf die örtlichen Gegebenheiten angepasst 
  werden.
  
  \section{Barrierefreie Spielplätze fördern}
  
  Ein barrierefreier Spielplatz muss die ganze Vielfalt aller Menschen 
  abdecken und ist grundsätzlich nicht sonderlich teurer als nicht   
  Barrierefreie. Man muss die Projekte nur von Anfang an richtig planen und 
  durchdenken. Spielplätze sind Begegnungsorte. Hier treffen sich Menschen 
  unterschiedlichen Alters, aus verschiedenen Gesellschaftsschichten, Menschen 
  mit und ohne Behinderung.
  
  Die Piratenpartei Augsburg setzt sich dafür ein, barrierefreie Spi"-elplätze 
  im Stadtgebiet zu fördern und zu fordern. Bereits bestehende Einrichtungen 
  sollen im Sinne der Barrierefreiheit und Inklusion sukzessive erweitert bzw. 
  saniert werden. Spielplätze für Rollstuhlfahrer, Spielgeräte die für 
  Rollstuhlfahrer nutzbar sind und Beschäftigungsmöglichkeiten, die von einem 
  Rollstuhl aus durchgeführt werden können, sollen hierbei geschaffen werden. 
  Für blinde oder sehbehinderte Spielplatzbesucher sollen Orientierungshilfen, 
  wie Leitlinien oder auffallend farblich gekennzeichnete Bereiche geschaffen 
  werden. Auf Geräten soll der Gleichgewichtssinn beansprucht und geschult 
  werden können. Anregungen für den Geruchssinn, den Hörsinn oder Tastsinn 
  sollen geschaffen werden. Soweit wie möglich sollen Spielplätze eine 
  nahegelegene (behindertengerechte) Toilette oder gar eine Wickelmöglichkeit 
  bieten. Ruhe- und Schattenplätze wären weiterhin erstrebenswert.
